\begin{table}[htb]
\ABNTEXfontereduzida
\caption[Funções do componente Class]{Funções do componente Class}
\label{tab-api-class}
\begin{tabular}{p{2.6cm}|p{2.6cm}|p{2.25cm}|p{6.8cm}}
%\hline
\textbf{Nome} & \textbf{Parâmetros} & \textbf{Retorno} & \textbf{Descrição} \\
\hline
\verb|__class| & - & table & Retorna a estrutura de classe do objeto em questão. \\
\hline
\verb|__classname| & - & string & Retorna o nome da classe do objeto. \\
\hline
\verb|__superClass| & - & metatable & Retorna a estrutura de classe da classe pai do objeto. \\
\hline
\verb|__is_a| & string: nomedaclasse & boolean & Verifica se o objeto é da mesma classe que a informada no parâmetro. Neste caso, verifica se os nomes são iguais. \\
\hline
\verb|__is_a| & table: classe & boolean & Verifica se o objeto é da mesma classe que a informada no parâmetro. Neste caso, verifica se a metatable é igual à metatabela de classe do objeto. \\
\hline
\verb|__respond-to| & string: método & boolean & Verifica se o método passado existe na classe em questão. Por ser uma tabela simulando uma classe, é possível que um objeto tenha mais funções visíveis do que a classe disponibiliza. Estas funções não são inspecionados por esta função. \\
% \hline
\end{tabular}
\end{table}